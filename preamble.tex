\usepackage[export]{adjustbox} % Nice alternative to minipage.
\usepackage{afterpage} % To give the title page its own geometry.
\usepackage[boxed,vlined,linesnumbered,resetcount,algosection]{algorithm2e} % For writing nice algorithms. 
\usepackage{amsmath} % Nice maths symbols.
\usepackage{amssymb} % Nice variable symbols.
\usepackage{amsthm} % Allows for nice definitions of theorem environments. 
\usepackage{array} % Allow for custom column widths in tables.
\usepackage{ltablex} % For long tables spanning multiple pages. % Must be before ARYDSHLN package!
\keepXColumns % Keeps the X column
\usepackage{arydshln} % Dashed lines using \hdashline \cdashline
%\usepackage[contents={}]{background} % Letting me add background overlays. 
\usepackage{bbm} % Gives Blackboard fonts. cf dsfont.
\usepackage{bm} % Bold math symbols.
\usepackage{calc} % Calculates widths of words. 
\usepackage{chngcntr} % Changing counters, e.g. with footnotes.
\usepackage{comment} % Allows large comment environments.
\usepackage{datetime} % For some  useful date time manipulation
\usepackage{datenumber}
\usepackage[]{draftwatermark} % Gives a draft overlay. Use options [nostamp] or [final].
\usepackage{dsfont} % Nicer blackboard fonts. use \mathds{...}
\usepackage{emptypage} % Empty pages have no headers and footers.
\usepackage{enumitem} % Nice listing options in itemize and enumerate.
\usepackage{esdiff} % Gives nice differential operators.
\usepackage{esvect} % Gives nice vector arrows.
\usepackage{etoolbox}
\usepackage{fancyhdr} % Nice headers.
\usepackage{fancyvrb} % Better verbatim, and verb|| allowed in footnotes. 
\usepackage{listings} % The listings package for code. %% Needs to be loaded before float. 
\usepackage{float} % Nice figure placement.
\usepackage[T1]{fontenc} % Nice range of text characters and accents.
\usepackage[bottom]{footmisc} % Nice footnote formatting.
\usepackage{graphicx} % Include figures.
\usepackage[notquote]{hanging} % For indenting later lines in a paragraph. USE the 'noquote' option else the `'` is overwritten, and breaks in maths mode!
\usepackage{ifoddpage} % Checks for odd or even page.
\usepackage{imakeidx} % Makes the index.
\usepackage{indentfirst} % Indents the first paragraph.
\usepackage{letltxmacro} % For defining a nice SQRT symbol.
\usepackage[mathlines]{lineno} % For numbering lines.
\usepackage{lipsum} % Useful for adding jargon.
\usepackage{lmodern} % Load a font with all the characters
\usepackage{makecell} % Break table headers with multiline entries.
\usepackage{marginnote} % For nice margin notes.
\usepackage{mathtools} % Gives the colon equals symbol.
\usepackage[framed,numbered,autolinebreaks,useliterate]{mcode} % Inports Listings package ideal for MATLAB.
\usepackage[framemethod=tikz]{mdframed} % Gives nices boxed and sidesrules.
\usepackage{mleftright} % For better left and right delimiters.
\usepackage{multirow} % Nice table cells spanning many rows.
\usepackage{multicol} % If I want to use multiple columns.
\usepackage{nameref}
\usepackage[numbers, sort&compress]{natbib} % Nice references.
\usepackage{nicefrac} % Gives nice fractions for superscripts.
\usepackage{nomencl} % Gives a symbol nomenclature. 
\usepackage[super]{nth} % Gives nice ordinal superscripts, eg 1st, 2nd, etc.
\usepackage{pifont} % For the check and cross marks
\usepackage{physics} % Nice partial derivatives and BRAKET notation.
\usepackage{ragged2e} % For nice allignment.
%\usepackage[norefs]{refcheck} % Can show any unused references.
\usepackage{romannum} % Nice typing for roman numerals.
\usepackage{rotating} % For sideways figures.
\usepackage[scr,scaled=1.1]{rsfso} % Gives Script fonts which are not so slanted. 
\usepackage{scalerel} % Allows the scaling of symbols. 
\usepackage{setspace} % Ideal for increasing line spacing. E.g.  \doublespacing
\usepackage[binary-units=true]{siunitx} % Nice formating of units.
\usepackage{sidenotes} % Nice margin figures and margin tables. 
\usepackage{subcaption} % Side by side figures.
\usepackage{thmtools}
\usepackage{tikz} % Nice diagrams.
\usepackage{titling} % Gives a bit of extra control around maketitle. 
\usepackage{tocbasic} % For better TOC alignment
\usepackage[nottoc]{tocbibind} % Gives nices Table of Contents
\usepackage[textsize=footnotesize]{todonotes} % A nice TODO list. [disable] to supress.
\usepackage{transparent} % Allows for transparent text. 
\usepackage[normalem]{ulem} % For striked through text.
\usepackage[hyphens]{url} % Allows urls to break on hyphens
\usepackage{usebib} % For citing a paper's title.
%\usepackage[table]{xcolor} % This is useful for making greyed table cells, nice for headers. Known preamble placement issues.
\usepackage{wasysym} % Gives some nice misc symbols, such as markers.
\usepackage{wrapfig}
\usepackage{xifthen}% Provides \isempty test.
\usepackage{xparse} % Gives \NewDocumentEnvironment which has nice optional argument handling.
\usepackage{xspace} % Gives nice spacing for commands.
%%%% Generally HYPERREF should be imported last. %%%%
\usepackage[colorlinks=true,linkcolor=black,urlcolor=black,citecolor=black,anchorcolor=black,breaklinks]{hyperref} % Colour links.
%%%% Should be loaded after hyperref. %%%%
\usepackage{cleveref} % Gives smart referencing. %% After Hyperref
\usepackage[margin=10pt,font=small,labelfont=bf,labelsep=endash,figurewithin=section,tablewithin=section]{caption} % Caption figures and tables nicely. %% After cleveref.
\usepackage[top=20mm,bottom=20mm,left=20mm, right=30mm, heightrounded, marginparwidth=24mm,  marginparsep=3mm, headsep=10mm]{geometry} % Use nice margins. Does give a small change in the default page margins. 
%[left=60mm,right=26mm,top=30mm,bottom=29mm, heightrounded, marginparwidth=41mm, marginparsep=8mm, headsep=10mm]


% Getting SI options setup.
\sisetup{range-phrase=--, range-units=single}


% Ensures subsubsections are numbered.
\setcounter{secnumdepth}{3}

% Makes math bold in headers and titles. 
\makeatletter
\g@addto@macro\bfseries{\boldmath}
\makeatother

\makeindex

\makenomenclature
% The column width for any nomenclature. 
\setlength\nomlabelwidth{0.2\linewidth}


% Set the table of content depth to only subsections. 
\setcounter{tocdepth}{2}


%\hbadness=10000 % Supresses bad box warnings


% Present the references in the order they are used.
%\bibliographystyle{unsrtnat} % Sorted by order of use.
%\bibliographystyle{plainnat} % Alphabetical. 
\bibliographystyle{customplainnat} % Alphabetical, with a more uniform formatting.
% Reduce spacing between references. 
\setlength{\bibsep}{0pt plus 0.3ex}
% For the usebib package to find the references. 
\bibinput{references}


% Listing -> Code in environment labels.
\renewcommand{\lstlistingname}{Code}
\renewcommand{\lstlistlistingname}{List of codes}
\crefname{listing}{code}{code}  
\Crefname{listing}{Code}{Codes}
\AtBeginDocument{\counterwithin{lstlisting}{section}} % Ensures these are numbered enough
\newfloat{lstfloat}{htbp}{lop} % environment for placing lisings in to make them float. 
% Make list of listings/code have the same inter-chapter spacing as other lists.
\let\Chapter\chapter
\def\chapter{\addtocontents{lol}{\protect\addvspace{10pt}}\Chapter}
\makeatletter
\patchcmd{\@chapter}{\chaptermark{#1}}{%
    \chaptermark{#1}%
    \addtocontents{lol}{\protect\addvspace{10\p@}}%
}{\typeout{Chapters patched for list-of-listings.}}{\typeout{Could not patch chapters for list-of-listings.}}
\makeatother

% List of Algorithms (NB, requires french spelling of 'algorithmes')
\renewcommand*{\listalgorithmcfname}{List of algorithms}
\newcommand{\listofalgorithmes}{\tocfile{\listalgorithmcfname}{loa}}



% Have theorems and lemmas be referenced by name if applicable.
\makeatletter
\renewrobustcmd{\cref}{\@osmcref{cref}}
\renewrobustcmd{\Cref}{\@osmcref{Cref}}
\def\@osmcref#1#2{%
    \begingroup
    \ifcsundef{r@#2}
    {}
    {\expandafter\expandafter\expandafter\expandafter\expandafter
        \expandafter\expandafter\def
        \expandafter\expandafter\expandafter\expandafter\expandafter
        \expandafter\expandafter\@osmcref@name
        \expandafter\expandafter\expandafter\expandafter\expandafter
        \expandafter\expandafter{%
            \expandafter\expandafter\expandafter
            \@thirdoffive\csname r@#2\endcsname}}%
    \ifcsundef{r@#2@cref}
    {}
    {\cref@gettype{#2}{\@osmcref@type}}%
    \ifboolexpr{not test {\ifdefvoid{\@osmcref@name}}
        and (test {\ifdefstring{\@osmcref@type}{theorem}}
        or test {\ifdefstring{\@osmcref@type}{lemma}}
        or test {\ifdefstring{\@osmcref@type}{corollary}})}
    {\nameref{#2} (\@cref{#1}{#2})}
    {\@cref{#1}{#2}}%
    \endgroup
}
\makeatother
% Give bold names to definitions and similar environments. 
\makeatletter
\def\th@plain{%
    \thm@notefont{}% same as heading font
    \itshape % body font
}
\def\th@definition{%
    \thm@notefont{}% same as heading font
    \normalfont % body font
}
\makeatother
% Giving correct theorem and lemma environments. 
\newtheorem{theorem}{Theorem}[section]
\newtheorem{corollary}{Corollary}[theorem]
\newtheorem{proposition}[theorem]{Proposition}
\newtheorem{lemma}[theorem]{Lemma}
\theoremstyle{definition}
\newtheorem{remark}{Remark}[theorem]
\newtheorem{definition}{Definition}[section]
\newtheorem{assumption}{Assumption}[section]
\newtheorem{proofof}{Proof}[theorem]
\AtEndEnvironment{proofof}{\qed}
% Naming these nicely.
\crefname{lemma}{lemma}{lemmas}
\Crefname{lemma}{Lemma}{Lemmas}
\crefname{theorem}{theorem}{theorems}
\Crefname{theorem}{Theorem}{Theorems}
\crefname{corollary}{corollary}{corollaries}
\Crefname{corollary}{Corollary}{Corollaries}
\Crefname{proposition}{proposition}{propositions}
\Crefname{proposition}{Proposition}{Propositions}
\crefname{remark}{remark}{remarks}
\Crefname{remark}{Remark}{Remarks}
\crefname{definition}{definition}{definitions}
\Crefname{definition}{Definition}{Definitions}
\crefname{assumption}{assumption}{assumptions}
\Crefname{assumption}{Assumption}{Assumptions}
% Change the end of proof symbol
\renewcommand\qedsymbol{\textbf{QED}}
% Making the proof in bold so it stands out more. 
\let\oldproofname=\proofname
\renewcommand{\proofname}{\rm\bf{\oldproofname}}

% Give bold names to definitions and similar environments. 
\makeatletter
\def\th@plain{%
    \thm@notefont{}% same as heading font
    \itshape % body font
}
\def\th@definition{%
    \thm@notefont{}% same as heading font
    \normalfont % body font
}
\makeatother

% Nice paragraph indents.
\setlength{\parindent}{5em}
\setlength{\parskip}{0.5\baselineskip}

% Giving the references the right title.
\renewcommand{\bibname}{References}
\renewcommand{\listfigurename}{List of figures}
\renewcommand{\listtablename}{List of tables}


% Removes hyphenation
\tolerance=1
\emergencystretch=\maxdimen
\hyphenpenalty=10000
\hbadness=10000

% To change the spacing in lists
%\setlist{noitemsep} % or \setlist{noitemsep} to leave space around whole list
%\setenumerate{itemsep=-0.4em,topsep=0.5em} % Seems to look nice.

% Custom column widths using C{2cm}, L, R, etc.
\newcolumntype{L}[1]{>{\raggedright\let\newline\\\arraybackslash\hspace{0pt}}m{#1}}
\newcolumntype{C}[1]{>{\centering\let\newline\\\arraybackslash\hspace{0pt}}m{#1}}
\newcolumntype{R}[1]{>{\raggedleft\let\newline\\\arraybackslash\hspace{0pt}}m{#1}}

% Gives a nice column separation in multicolumn mode.
\setlength{\columnsep}{5mm}

% Figure environment for use in multicolumn. To put in captions use \captionof{figure}{content of caption}.
\newenvironment{Figure}
{\par\medskip\noindent\minipage{\linewidth}}
{\endminipage\par\medskip}

% Gives the nice SQRT symbol.
\makeatletter
\let\oldr@@t\r@@t
\def\r@@t#1#2{%
	\setbox0=\hbox{$\oldr@@t#1{#2\,}$}\dimen0=\ht0
	\advance\dimen0-0.2\ht0
	\setbox2=\hbox{\vrule height\ht0 depth -\dimen0}%
	{\box0\lower0.4pt\box2}}
\LetLtxMacro{\oldsqrt}{\sqrt}
\renewcommand*{\sqrt}[2][\ ]{\oldsqrt[#1]{#2}}
\makeatother

\DeclareMathOperator{\sign}{sign}
\DeclareMathOperator*{\argmin}{argmin}
\DeclareMathOperator*{\argmax}{argmax}
\newcommand{\indicatorfn}{\mathds{1}}
\newcommand{\iconst}{\mathrm{i}}
\newcommand{\econst}{\mathrm{e}}
\newcommand{\erfinv}{\mathrm{inverf}}
\newcommand{\inverf}{\erfinv}

% A nice conditional expectation environment for use with probability spaces. 
% Based on Physics package. 
% \conExp[Measure][Subset]{Argument}{Condition}   (Condition is optional field)
\DeclareDocumentCommand\conExp{s o o m g}{
    % s : star, to use non scaling parenthesese.
    % o : optional, probability measure.
    % o : optional, set over which expectation is taken.
    % m : mandatory, argument.
    % g : optional, condition.
    \IfBooleanTF{#1}{}{\trigbraces}{\mathbb{E}\IfNoValueTF{#2}{}{^{#2}}\IfNoValueTF{#3}{}{_{#3}}}(#4\IfNoValueTF{#5}{}{\IfBooleanTF{#1}{\mid}{\,\middle|\,}#5})
}

\DeclareDocumentCommand\conVar{s o o m g}{
    % s : star, to use non scaling parenthesese.
    % o : optional, probability measure.
    % o : optional, set over which expectation is taken.
    % m : mandatory, argument.
    % g : optional, condition.
    \IfBooleanTF{#1}{}{\trigbraces}{\mathbb{V}\IfNoValueTF{#2}{}{^{#2}}\IfNoValueTF{#3}{}{_{#3}}}(#4\IfNoValueTF{#5}{}{\IfBooleanTF{#1}{\mid}{\,\middle|\,}#5})
}

\DeclareDocumentCommand\quadVar{s m g g}{
    % s : star, to use non scaling parenthesese.
    % m : mandatory, first argument.
    % g : optional, evaluation time.
    % g : optional, second argument.
    \IfBooleanTF{#1}{\expval*{#2\IfNoValueF{#4}{,#4}}\IfNoValueF{#3}{_{#3}}}{\expval{#2\IfNoValueF{#4}{,#4}}\IfNoValueF{#3}{_{#3}}}
}

\newlength{\LowercaseLength}
\AtBeginDocument{\settoheight{\LowercaseLength}{a}} % after \normalsize
\DeclareDocumentCommand\lorder{s m}{%
    \IfBooleanTF{#1}{}{\trigbraces}{\resizebox{!}{\LowercaseLength}{$\mathcal{O}$}}(#2)%
}


% Number equations down to the subection level, e.g. 1.2.3 is the third equation in
% subsection 2 of section 1.
\numberwithin{equation}{section}
\newcommand*\tageq{\refstepcounter{equation}\tag{\theequation}}
% Ensure equations are correctly formatted with cleverref
\crefname{equation}{}{}
\creflabelformat{equation}{(#2#1#3)}


% This makes the footnote counter reset in each section.
\counterwithin*{footnote}{section}

% Nice spacing in the first fow of a table
\newcommand{\firstrowspacing}{\rule{0pt}{2.6ex}}
% For a more open look in tables.
\setlength\extrarowheight{3pt} 

% Some corporate names we may need to append a \textsuperscript{\textregistered} into. 
\newcommand{\nag}{NAG\xspace}
\newcommand{\arm}{Arm\xspace}
\newcommand{\intel}{Intel\xspace}
\newcommand{\openmp}{OpenMP\xspace}
\newcommand{\amd}{AMD\xspace}
\newcommand{\ibm}{IBM\xspace}
\newcommand{\nvidia}{Nvidia\xspace}
\newcommand{\icdf}{inverse cumulative density function\xspace}
\newcommand{\rng}{random number generator\xspace}
\newcommand{\PhiInverse}{$ \Phi^{-1}(\cdot) $\xspace}
\newcommand{\Ito}{It\^{o}\xspace}
\newcommand{\BDG}{Burkholder-Davis-Gundy\xspace}
\newcommand{\CIR}{Cox-Ingersoll-Ross\xspace}


% The nice headers and footers.
\pagestyle{fancy}
% Modifying the header and footer styles for chapters and sections. 
\renewcommand{\sectionmark}[1]{\markboth{#1}{#1}}

% Specifying the headers and footers. 
\fancyhf{}
\renewcommand*{\headrulewidth}{1pt}%
\fancyhead[LO]{\nouppercase{\leftmark}}
\fancyhead[RE]{\nouppercase{\rightmark}}
\fancyhead[LE,RO]{\thepage}
\fancypagestyle{plain}{%
	\fancyhf{}%
	\renewcommand*{\headrulewidth}{0pt}%
}
\cfoot{}%\thepage}





% Make margin notes small
\renewcommand*{\marginfont}{\noindent \footnotesize}
%\reversemarginpar % If I want a thick margin by the binding.
\renewcommand\raggedrightmarginnote{\sloppy}
\renewcommand\raggedleftmarginnote{\sloppy}


% Gives a nice quote environment.
\NewDocumentEnvironment{myquote}{O{}}{%
	\begin{center}
		\begin{minipage}{0.85\linewidth}
			\vspace{1ex}
			\centering \itshape \justifying}
		{%
			\ifthenelse{\isempty{#1}}{}{
			\begin{flushright}%The author/source.
				\normalfont #1
			\end{flushright}}
						\vspace{1ex}
		\end{minipage}
	\end{center}
}

% Gives a nice siderule environment. e.g. \begin{siderules}
\newmdenv[topline=false,bottomline=false,rightline=false,skipabove=\topsep,skipbelow=\topsep,backgroundcolor=none]{siderules}

% For a numbered and description. Use inside enumerate, \litem{Something} etc. 
\newcommand\litem[1]{\item{\textit{#1} \\ \hfill \vspace*{-1.5ex} \\ \indent}}

% Nice spacing in lists
%\setlist{listparindent=\parindent,parsep=1ex} 

% This alligns figures in the adjust box environment to the inner margin. 
\newcommand{\aligninner}{\ifoddpage \raggedright \else \raggedleft \fi}%


% Gives a nice aligned figure environment, where figures are flush to the inner margin overflowing off the outer margin first. Useful for very wide figures, or set of lots of fub figures. 
\NewDocumentEnvironment{myalignedfigure}{O{1.3} O{htb}}{% fractional_linewidth,  position
	\begin{figure}[#2]
	\checkoddpage
	\edef\whichside{\ifoddpage left\else right\fi}
		\begin{adjustbox}{\whichside, minipage=#1\linewidth}}
		{%
		\end{adjustbox}
	\end{figure}
	}

% Gives a nice draft text.
\SetWatermarkScale{1}
\SetWatermarkLightness{0.9}

% The oxford comma from cref for multiple citations. 
\newcommand{\creflastconjunction}{, and\nobreakspace}

% Allows for bold face typed text fonts for computer modern font family. Useful in listings. 
\DeclareFontShape{OT1}{cmtt}{bx}{n}{<5><6><7><8><9><10><10.95><12><14.4><17.28><20.74><24.88>cmttb10}{}



% Allowing alignat to have repeating alignment options. 
\makeatletter
\def\align@preamble@r{%
    \hfil
    \strut@
    \setboxz@h{\@lign$\m@th\displaystyle{##}$}%
    \ifmeasuring@\savefieldlength@\fi
    \set@field
    \tabskip\z@skip}
\def\align@preamble@l{%
    \strut@
    \setboxz@h{\@lign$\m@th\displaystyle{{}##}$}%
    \ifmeasuring@\savefieldlength@\fi
    \set@field
    \hfil
    \tabskip\alignsep@}
\def\align@preamble@c{%
    \hfil
    \strut@
    \setboxz@h{\@lign$\m@th\displaystyle{{}##}$}%
    \ifmeasuring@\savefieldlength@\fi
    \set@field
    \hfil
    \tabskip\alignsep@
}
\def\setalignpreamble#1{\def\align@preamble{}\x@setalignpreamble#1\relax}
\def\x@setalignpreamble#1{%
    \ifx\relax#1\else
    \edef\align@preamble{%
        \expandafter\unexpanded\expandafter{\align@preamble}%
        &\span\expandafter\noexpand\csname align@preamble@#1\endcsname}%
    \expandafter\x@setalignpreamble
    \fi}


% A more extreme version of shortintertext
\newcommand\prefixtext[1]{%
    \ifvmode\else\\\@empty\fi
    \noalign{%
        \penalty0%
        \vbox{\mathstrut}%
        \penalty10000%
        \vskip-\baselineskip
        \penalty10000%
        \vbox to 0pt{%
            \normalbaselines
            \ifdim\linewidth=\columnwidth
            \else
            \parshape\@ne
            \@totalleftmargin\linewidth
            \fi
            \vss
            \noindent#1\par}%
        \penalty10000%
        \vskip-\baselineskip}%
    \penalty10000}


% Gives a nice thousand separator. 
\sisetup{group-separator = \text{\,}, group-minimum-digits=5}

% Makes thead the normal font size. 
\renewcommand\theadfont{}

% \fp[precision]{Argument} (Condition is optional field)
\DeclareDocumentCommand\fp{o m}{
    % s : star, to use non scaling parenthesese.
    % o : optional, precision level.
    % m : mandatory, argument.
    \bar{#2}\IfNoValueTF{#1}{}{^{(#1)}}
}
% \fpSQRT[precision]{Argument} 
\DeclareDocumentCommand\fpSQRT{s o m}{
    % s : star, to use non scaling parenthesese.
    % o : optional, precision level.
    % m : mandatory, argument.
    \mathinner{\IfBooleanTF{#1}{}{\trigbraces}{\overline{\mathrm{sqrt}}\IfNoValueTF{#2}{}{^{(#2)}}}(#3)}
}
\DeclareDocumentCommand\fpRound{s o m}{
    % s : star, to use non scaling parenthesese.
    % o : optional, precision level.
    % m : mandatory, argument.
    \mathinner{\IfBooleanTF{#1}{}{\trigbraces}{\mathscr{R}}(#3\IfNoValueTF{#2}{}{;{#2}})}
}
\DeclareDocumentCommand\support{s m}{
    % s : star, to use non scaling parenthesese.
    % m : mandatory, argument.
    \mathinner{\IfBooleanTF{#1}{}{\trigbraces}{\mathrm{supp}}(#2)}
}

% Making some circled operators. (use \ocircle for a smaller circle)
\makeatletter
\newcommand\incircbin
{%
  \mathpalette\@incircbin
}
\newcommand\@incircbin[2]
{%
  \mathbin%
  {%
    \ooalign{\hidewidth$#1#2$\hidewidth\crcr$#1\bigcirc$}%
  }%
}
\newcommand{\fpPlus}{\incircbin{+}}
\newcommand{\fpMinus}{\incircbin{-}}
\newcommand{\fpTimes}{\incircbin{\times}}
\newcommand{\fpDivide}{\incircbin{\divisionsymbol}}
\newcommand{\fpOp}{\incircbin{\ast}}
\makeatother

%\newcommand{\fpPlus}{\oplus}
%\newcommand{\fpMinus}{\ominus}
%\newcommand{\fpTimes}{\otimes}
%\newcommand{\fpDivide}{\oslash} 
%\newcommand{\fpOp}{\circledast}


%\newcommand{\sumfine}{\mathop{{{\sum}^{\mathrlap{\,\mathrm{f}}}}}}
%\newcommand{\sumcoarse}{\mathop{{{\sum}^{\mathrlap{\,\mathrm{c}}}}}}
\DeclareDocumentCommand\sumfine{g g}{
    % g : optional, subscript.
    % g : optional, superscript.
    \mathop{{{\sum}_{\mathrlap{\mathrm{f}}}}}\limits\IfNoValueF{#1}{_{#1}}\IfNoValueF{#2}{^{#2}}\,
}
\DeclareDocumentCommand\sumcoarse{g g}{
    % g : optional, subscript.
    % g : optional, superscript.
    \mathop{{{\sum}_{\mathrlap{\mathrm{c}}}}}\limits\IfNoValueF{#1}{_{#1}}\IfNoValueF{#2}{^{#2}}\,
}

\DeclareDocumentCommand\supfine{g}{
    % g : optional, subscript.
    \mathop{\mathrm{sup}^{\mathrlap{\mathrm{f}}}}\IfNoValueF{#1}{_{#1}}\,
}
\DeclareDocumentCommand\supcoarse{g}{
    % g : optional, subscript.
    \mathop{\mathrm{sup}^{\mathrlap{\mathrm{c}}}}\IfNoValueF{#1}{_{#1}}\,
}



% For usage in align environments. 
\newcommand\phantomrel[1]{\mathrel{\phantom{#1}}}
\newcommand\phantombin[1]{\mathbin{\phantom{#1}}}

% For defining some paired delimited items, such as set, floor, ceil, etc, where the spacing is correct and scaling is supressed with a * notation, e.g. \set{} and \set*{}
\NewDocumentCommand\xDeclarePairedDelimiter{mmm}
{%
    \NewDocumentCommand#1{som}{%
        \IfNoValueTF{##2}
        {\IfBooleanTF{##1}{#2##3#3}{\mleft#2##3\mright#3}}
        {\mathopen{##2#2}##3\mathclose{##2#3}}%
    }%
}
\xDeclarePairedDelimiter{\set}{\lbrace}{\rbrace}
\xDeclarePairedDelimiter{\floor}{\lfloor}{\rfloor}
\xDeclarePairedDelimiter{\ceil}{\lceil}{\rceil}
\xDeclarePairedDelimiter{\parens}{(}{)} 

\DeclareDocumentCommand\logtwo{}{\trigbraces{{\logarithm}_{2}}}
\DeclareDocumentCommand\logten{}{\trigbraces{{\logarithm}_{10}}}

% Allowing verbatim in footnotes. 
\VerbatimFootnotes

\newcommand{\cmark}{\ding{51}}%
\newcommand{\xmark}{\ding{55}}%

% Making a 4-way difference operator (looks like the D'Alembertian)
\makeatletter
\newcommand{\fourdiff}{{\mathpalette\fourdiff@\relax}}%{\mathop{\mathpalette\fourdiff@\relax}}
\newcommand{\fourdiff@}[2]{%
  \begingroup
  \sbox\z@{$\m@th#1\square$}%
  \dimen0=\fontdimen8
    \ifx#1\displaystyle\textfont\else
    \ifx#1\textstyle\textfont\else
    \ifx#1\scriptstyle\scriptfont\else
    \scriptscriptfont\fi\fi\fi3
  \makebox[\wd\z@]{%
    \hbox to \ht\z@{%
      \vrule width \dimen0
      \kern-\dimen0
      \vbox to \ht\z@{
        \hrule height \dimen0 width \ht\z@
        \vss
        \hrule height 2\dimen0
      }%
      \kern-2.5\dimen0
      \vrule width 2.5\dimen0
    }%
  }%
  \endgroup
}
\makeatother



\newcommand{\contradiction}{\scaleobj{1.5}{\text{\textreferencemark}}}


% So Roman page numbers are right aligned. and proper spacing between section numbers and captions
\DeclareTOCStyleEntry[numwidth=3.5em]{tocline}{figure}% for figure entries
\DeclareTOCStyleEntries[
  level:=figure,
  indent:=figure,
  numwidth:=figure,
  dynnumwidth% enlarges numwidth automatically if the entry number is too long; needs an additional run
]{tocline}{table,lstlisting,algocf,todo}
\DeclareTOCStyleEntries[
  rightindent=10em,
  pagenumberbox=\pagenumberbox
]{tocline}{section,subsection,subsubsection,paragraph,subparagraph,figure,table,lstlisting,algocf,todo}
\newcommand*\pagenumberbox[1]{\mbox{\hspace{0em}#1}}



%%%% Patching some math environments from AMS for better line numbering. %%%%

%% Patch 'normal' math environment: (currently unused, but good to have)
% \newcommand*\linenomathpatch[1]{%
%   \expandafter\pretocmd\csname #1\endcsname {\linenomath}{}{}%
%   \expandafter\pretocmd\csname #1*\endcsname{\linenomath}{}{}%
%   \expandafter\apptocmd\csname end#1\endcsname {\endlinenomath}{}{}%
%   \expandafter\apptocmd\csname end#1*\endcsname{\endlinenomath}{}{}%
% }
%% Patch AMS math environment:
\newcommand*\linenomathpatchAMS[1]{%
    \expandafter\pretocmd\csname #1\endcsname {\linenomathAMS}{}{}%
    \expandafter\pretocmd\csname #1*\endcsname{\linenomathAMS}{}{}%
    \expandafter\apptocmd\csname end#1\endcsname {\endlinenomath}{}{}%
    \expandafter\apptocmd\csname end#1*\endcsname{\endlinenomath}{}{}%
}

%% Definition of \linenomathAMS depends on whether the mathlines option is provided
\expandafter\ifx\linenomath\linenomathWithnumbers
\let\linenomathAMS\linenomathWithnumbers
%% The following line gets rid of an extra line numbers at the bottom:
\patchcmd\linenomathAMS{\advance\postdisplaypenalty\linenopenalty}{}{}{}
\else
\let\linenomathAMS\linenomathNonumbers
\fi

% \linenomathpatch{equation} %% <- unnecessary, equation is already patched
\linenomathpatchAMS{gather}
\linenomathpatchAMS{multline}
\linenomathpatchAMS{align}
\linenomathpatchAMS{alignat}
\linenomathpatchAMS{flalign}

% Mark the style as style=C and it will highlight more keywords. 
\lstdefinestyle{C}{
    language=C,
    morekeywords={omp,simd,reduction,simdlen,declare,inline,bool,restrict,half},
    otherkeywords={\#pragma,\_\_fp16,\#if,\#else,\#endif,\#ifdef}
}

% Making an independent symbol.
\newcommand\independent{\protect\mathpalette{\protect\independenT}{\perp}}
\def\independenT#1#2{\mathrel{\rlap{$#1#2$}\mkern2mu{#1#2}}}

% Increase algorithm spacing between box and caption and font size.
\SetAlCapSkip{1em}
\SetAlCapNameFnt{\small}
\SetAlCapFnt{\small}
